\documentclass[../notes.tex]{subfiles}

\pagestyle{main}
\renewcommand{\chaptermark}[1]{\markboth{\chaptername\ \thechapter\ (#1)}{}}
\setcounter{chapter}{4}

\begin{document}




\chapter{Magnetic Resonance Spectroscopy}
\section{Lecture 8: Magnetic Resonance Spectroscopy}
\begin{itemize}
    \item \marginnote{1/31:}Refer to Chapter 14 of \textcite{bib:McQuarrieSimon}.
    \item There is both NMR (nuclear magnetic resonance) and ESR (electron spin resonance) or EPR (electron paramagnetic resonance).
    \begin{itemize}
        \item The last two are the same thing.
    \end{itemize}
    \item Two fields: The static magnetic field, and the probing \emph{electro}magnetic field.
    \item Derivation of quantized angular momentum.
    \begin{itemize}
        \item In molecules, there is a multiplicity/degeneracy of states that grows as $2J+1$. They have quantized angular momentum.
        \item So is the orbital angular momentum of electrons!
        \item Putting atoms into a magnetic field \emph{creates} the anisotropy necessary for discussing the $z$-component (or any coordinate component) of angular momentum.
        \item Spin angular momentum: Just means that the objects (e.g., electrons and nuclei) have a property that looks a lot like spin and/or angular momentum.
        \item We say that each nucleon has a spin of $1/2$. Protons and neutrons add separately.
        \begin{itemize}
            \item Even number of protons and neutrons? Spin 0.
            \item Mixed even/odd? We have actual nuclear spin.
            \begin{itemize}
                \item We need a nucleus like this to detect!
            \end{itemize}
            \item Odd/odd? We have 0 spin again.
        \end{itemize}
        \item We focus on spin $1/2$ particles. These have two degenerate energy states that split in a magnetic field.
    \end{itemize}
    \item Classical picture of spin angular momentum.
    \begin{itemize}
        \item Picture a charged particle with angular momentum. The circulating charge produced a magnetic field which aligns along the direction of the angular momentum.
        \item Indeed, a spinning charged particle behaves like a dipole.
        \item $\gamma=2mc$ is the \textbf{gyromagnetic ratio}.
    \end{itemize}
    \item Quantum spin angular momentum.
    \begin{itemize}
        \item Basically the same thing; we just rephrase everything from before in the language of operators.
    \end{itemize}
    \item \textbf{Zeeman effect}: Two energy levels split with increasing $B$.
    \item \textbf{Larmor frequency}: The frequency $\nu=\gamma B/2\pi$ in the radio frequency range that induces a shift.
    \item Typical operating conditions.
    \begin{itemize}
        \item NMR vs. ESR: NMR has a stronger magnetic field, longer EM excitation radio waves, and significantly lower gyromagnetic ratios.
        \item There is only a tiny difference between nuclear state occupation at room temperature; hence supercooling to get something detectable.
    \end{itemize}
    \item What is the magnetic dipole doing in the magnetic field?
    \begin{itemize}
        \item No constraints on the $x$ and $y$ components of $I$.
        \item A dipole in a field experiences a torque.
        \begin{itemize}
            \item Dipole precesses around $B$ at the Larmor frequency $\nu$.
        \end{itemize}
        \item FT-NMR spectrometers use pulsed rf fields to synchronize and detect the procession of spins.
    \end{itemize}
    \item Lots of good extension material on NMR; also worth rewatching at some point!
\end{itemize}



\section{Lecture 9: Magnetic Resonance Spectroscopy 2}
\begin{itemize}
    \item \marginnote{2/2:}Summary of last time.
    \begin{itemize}
        \item The quantity that we're measuring is spin angular momentum $\bar{I}$, which is a vector quantity.
        \begin{equation*}
            |\bar{I}| = \hbar\sqrt{I(I+1)}
        \end{equation*}
        where $I=1/2$ is the nuclear spin quantum number.
        \item The other quantity of concern is the projection $I_z=m_I\hbar$ where $m_i=\pm 1/2$.
        \item In a magnetic field, we break degeneracy, getting $E(m_I)=-\gamma_N\hbar m_IB$ and $\Delta E=-\gamma_N\hbar B$.
        \item Electromagnetic resonance is achieved when the frequency $\nu$ of incident radiation satisfies $h\nu=\Delta E$.
    \end{itemize}
    \item The interest in chemistry: Chemical shift.
    \begin{itemize}
        \item There are small variations in the frequency for different types of protons depending on the surrounding electron density.
        \item Measured frequency depends on effective magnetic field.
        \begin{itemize}
            \item Shielding: The influence of electrons around the nucleus on the effective magnetic field.
            \item The effective field is smaller than the applied field.
        \end{itemize}
        \item Shielding \emph{decreases} the splitting (this is why nearby highly polar groups lead to large shifts, while alkanes have small shifts).
    \end{itemize}
    \item Measuring the chemical shift.
    \begin{itemize}
        \item We measure the difference in the Larmor frequency relative to a standard (TMS).
        \item Shielding is a small effect (on the order of \num{e-6}, so we use ppm $\delta$).
        \item Example: 1 ppm at \SI{500}{\mega\hertz} is $\nu=\SI{500}{\hertz}$, which is tiny (on the order of microjoules).
    \end{itemize}
    \item Chemical shift charts (from OChem) are included in the slides.
    \item FT-NMR spectrometers.
    \begin{itemize}
        \item How do we make these measurements?
        \item NMR spectrometers are almost all working in FT mode these days.
        \item They use pulsed radiofrequency (r.f.) fields and detect the precession of spins.
        \item Precession of one spin in a magnetic field occurs at the Larmor frequency.
        \item Applying an excitation field creates a superposition of $m_s=\pm 1/2$ states. The net dipole is now perpendicular, and precesses that way (i.e., in the $xy$ plane) in a mathematically describable fashion.
        \item Putting your superconducting coil along the $x$- or $y$-axis allows you to detect changes.
    \end{itemize}
    \item \textbf{Magnetization}: The macroscopic alignment of magnetic dipoles $\bar{M}=\sum\bar{\mu}$.
    \begin{itemize}
        \item At equilibrium, $\bar{M}$ aligns along $\bar{B}$.
        \item An rf field rotates magnetization to $x$: This is a \textbf{$\bm{\pi/2}$-pulse} or a \textbf{$\bm{90^\circ}$-pulse}.
        \item We detect precessing magnetization and return to equilibrium during the \textbf{free-induction decay}. An FT of the decay then generates our spectrum.
    \end{itemize}
    \item Relaxation mechanism.
    \begin{itemize}
        \item Spin state lifetime ($T_1$).
        \begin{itemize}
            \item "Spin-lattice" or "longitudinal" relaxation.
            \item Recovery of the magnetization along $z$.
            \item A molecular property.
            \item Transfer of energy to the environment.
            \item The return of magnetization to equilibrium has a characteristic time constant $T_1$ which appears in the time vs. relaxation plot $1-\e[-t/T_1]$.
        \end{itemize}
        \item Dephasing ($T_2$).
        \begin{itemize}
            \item "Transverse" relaxation.
            \item Loss of magnetization in the $xy$-plane of many different sources.
            \item We have the loss described by $\e[-t/T_2]$.
        \end{itemize}
        \item These two processes are not independent.
    \end{itemize}
    \item There are numerous types of NMR experiments.
    \begin{itemize}
        \item In our lab, we just scratch the surface.
        \item We use a population inversion to measure $T_1$.
        \begin{itemize}
            \item \textbf{$\bm{\pi}$-pulse} to invert magnetization.
            \item Wait for relaxation.
            \item Read out following another \textbf{$\bm{\pi/2}$-pulse}.
        \end{itemize}
        \item Two-dimensional spectroscopy.
        \begin{itemize}
            \item Heteronuclear single quantum coherence spectroscopy (HSQC).
            \item Excite \ce{{}^1H}; transfer its magnetization to \ce{{}^13C}, which is nice because \ce{{}^13C} is hard to excite on its own.
            \item Transfer back to \ce{{}^1H} and detect.
            \item Tells us which protons transfer magnetization.
        \end{itemize}
    \end{itemize}
\end{itemize}




\end{document}