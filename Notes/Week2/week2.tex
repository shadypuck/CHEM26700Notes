\documentclass[../notes.tex]{subfiles}

\pagestyle{main}
\renewcommand{\chaptermark}[1]{\markboth{\chaptername\ \thechapter\ (#1)}{}}
\stepcounter{chapter}

\begin{document}




\chapter{???}
\section{Lecture 3: Time, Frequency, and Fourier Transforms}
\begin{itemize}
    \item \marginnote{1/10:}Frequency- and time-domain spectroscopy.
    \begin{itemize}
        \item Two ways of extracting the same information.
        \begin{enumerate}
            \item Absorption spectrum (frequency domain).
            \begin{itemize}
                \item Vary frequency of driving field or disperse white light after passing through the sample and look at each frequency component.
                \item Measure the power absorbed for different frequencies.
                \item Tells us the resonance frequency, how strongly the light interacts with the matter, and damping times or relaxation processes.
            \end{itemize}
            \item Pulsed excitation (time domain).
            \begin{itemize}
                \item Apply a pulsed driving force.
                \item Measure resultant periodic oscillation and relaxation.
                \item This is the basis for modern NMR and FTIR instruments.
            \end{itemize}
        \end{enumerate}
        \item Both represent the time-dependent behavior of a molecule.
    \end{itemize}
    \item A powerful reason to use the time domain is the formal relationship between time and frequency data, the \textbf{Fourier transform}.
    \item \textbf{Fourier transform}: A formal relationship between the time domain and the frequency domain.
    \begin{itemize}
        \item Underlying idea: Any function can be expressed as a sum of sines and cosines, i.e.,
        \begin{equation*}
            F(t) = \sum_{n=1}^\infty[a_n\cos(n\bar{\omega}t)+b_n\sin(n\bar{\omega}t)]
        \end{equation*}
        \item In practice, sample $N$ points over a period $T$.
        \begin{itemize}
            \item The values of time at which we sample are $t=n\delta t$ for $n=1,\dots,m$.
        \end{itemize}
        \item Numerical analysis: Expand in harmonics of base frequency, i.e., we let $\bar{\omega}=\pi/T$ be 1/2 cycle of $T$. The harmonics are $n\bar{\omega}$ for $n=1$ to $N$. We thus have as many harmonics as we do data points.
        \item Then we plot the expansion coefficients vs. the frequency, and that is the spectrum in the series of expansion coefficients.
        \item There's more to it than this, but this is the basic concept.
        \item Also, this is only a discrete data set.
    \end{itemize}
    \item \textbf{Fourier analysis}: Determining the coefficients $a_n,b_n$.
    \item Fourier transform relations.
    \begin{itemize}
        \item For continuous functions, we use Fourier transform integrals.
        \begin{itemize}
            \item Note that although both of the following integrals are sine transforms, there also exist cosine and complex $\e[-i\omega t]$ transforms.
            \item Sine and cosine transforms are used for real data; the complex form is more general.
        \end{itemize}
        \item To convert to the time domain $S(t)$, we write
        \begin{equation*}
            S(t) = \frac{1}{\sqrt{2\pi}}\int_{-\infty}^{+\infty}\tilde{S}(\omega)\sin\omega t\dd\omega
        \end{equation*}
        \item To convert to the frequency domain $\tilde{S}(\omega)$, we write
        \begin{equation*}
            \tilde{S}(\omega) = \frac{1}{\sqrt{2\pi}}\int_{-\infty}^{+\infty}S(t)\sin\omega t\dd{t}
        \end{equation*}
        \item Example: Damped harmonic oscillator.
        \begin{equation*}
            S(t) \propto \e[-\gamma t]\sin\omega_0t
            \quad\Longleftrightarrow\quad
            \tilde{S}(\omega) \propto \frac{\gamma}{(\omega-\omega_0)^2+\gamma^2}
        \end{equation*}
    \end{itemize}
    \item Parameters in the time and frequency domains.
    \begin{figure}[h!]
        \centering
        \begin{tikzpicture}
            \footnotesize
            \begin{scope}
                \draw [-latex] (0,0) -- (5,0) node[below left]{$t$};
                \node at (-1,2.75) {\small$S(t)$};
    
                \draw [blx,yshift=4.7cm,yscale=0.5] plot[domain=0:5,samples=1000,smooth,/pgf/fpu,/pgf/fpu/output format=fixed] (\x,{e^(-0.5*\x)*sin(50*\x r)});
                \draw [grx,yshift=3.4cm,yscale=0.5] plot[domain=0:5,samples=1000,smooth,/pgf/fpu,/pgf/fpu/output format=fixed] (\x,{e^(-1.5*\x)*sin(50*\x r)});
                \draw [orx,yshift=2.1cm,yscale=0.5] plot[domain=0:5,samples=100,smooth,/pgf/fpu,/pgf/fpu/output format=fixed] (\x,{e^(-0.5*\x)*sin(20*\x r)});
                \draw [rex,yshift=0.8cm,yscale=0.5] plot[domain=0:5,samples=100,smooth,/pgf/fpu,/pgf/fpu/output format=fixed] (\x,{e^(-1.5*\x)*sin(20*\x r)});
            \end{scope}
            \begin{scope}[xshift=7cm]
                \draw [-latex] (0,0) -- (5,0) node[below left]{$\omega$};
                \node at (6,2.75) {\small$\tilde{S}(\omega)$};
    
                \draw [blx,xshift=2.5cm,yshift=4.7cm,yscale=0.5] plot[domain=-2.5:2.5,samples=500,smooth] (\x,{0.1/(18*(\x*\x+1/324))});
                \draw [grx,xshift=2.5cm,yshift=3.4cm,yscale=0.5] plot[domain=-2.5:2.5,samples=500,smooth] (\x,{0.1/(9*(\x*\x+1/81))});
                \draw [orx,xshift=1.5cm,yshift=2.1cm,yscale=0.5] plot[domain=-1.5:3.5,samples=500,smooth] (\x,{0.1/(18*(\x*\x+1/324))});
                \draw [rex,xshift=1.5cm,yshift=0.8cm,yscale=0.5] plot[domain=-1.5:3.5,samples=500,smooth] (\x,{0.1/(9*(\x*\x+1/81))});
            \end{scope}
        \end{tikzpicture}
        \caption{Parameters in the time and frequency domains.}
        \label{fig:paramTimeFreq}
    \end{figure}
    \begin{table}[h!]
        \centering
        \small
        \renewcommand{\arraystretch}{1.2}
        \begin{tabular}{l|l|l}
            \textbf{Parameter} & \textbf{Time Domain $\bm{S(t)}$} & \textbf{Frequency Domain $\bm{\tilde{S}(\omega)}$}\\
            \hline
            Large $\omega_0$ & Fast oscillations & High frequency\\
            Small $\omega_0$ & Slow oscillations & Low frequency\\
            Large $\gamma$ & Fast decay & Broad linewidth\\
            Small $\gamma$ & Slow decay & Narrow linewidth\\
        \end{tabular}
        \caption{Parameters in the time and frequency domains.}
        \label{tab:paramTimeFreq}
    \end{table}
    \begin{itemize}
        \item The area under the lines remains constant.
        \item Notice how the lower-frequency waves (orange and red) have frequency spikes shifted down.
        \item Broader Lorentzian implies more different types of frequencies are present implies destructive interference takes hold more quickly implies quicker decay.
    \end{itemize}
    \item F.T. Example: Two resonances.
    \begin{itemize}
        \item Consider a superposition of two oscillating decaying functions
        \begin{equation*}
            C(t) = \e[-\gamma t]\sin(\omega_1t)+\e[-\gamma t]\sin(\omega_2t)
        \end{equation*}
        \item This implies two resonances in the spectrum.
        \begin{itemize}
            \item In particular, they manifest as two beat frequencies, one of which is the average frequency, and the other of which is the difference.
            \item The average frequency determines the regular vibrations; the difference is the bounding function.
        \end{itemize}
    \end{itemize}
    \item History of science during the French revolution.
    \begin{itemize}
        \item Lavoisier and Fourier were both strongly influenced by their time (the French revolution).
        \item Lavoisier was an elite tax collector, and was sentenced to be executed. When he asked the judge for mercy, the judge said, "the Republic has no need for scientists."
        \item Fourier got into trouble with Robespierre even though he was a revolutionary, but Robespierre's regime was overthrown a day before his scheduled execution. Thus, we get Fourier transforms!
    \end{itemize}
    \item Fourier transform infrared spectrometer.
    \begin{itemize}
        \item Michelson Interferometer.
        \begin{itemize}
            \item Named after UChicago's first physics chair, also the first American to be awarded the Nobel prize in physics.
        \end{itemize}
        \item How it works: Intensity changes with pathlength-induced interference.
        \begin{itemize}
            \item Incoming monochromatic waves get half reflected, half transmitted, allowing for phase separation.
            \item Mathematically,
            \begin{align*}
                \Delta L &= \frac{1}{\bar{\nu}} = \frac{c}{\nu}&
                \Delta t &= \frac{\Delta L}{c}
            \end{align*}
            \item Thus, if our light source is emitting monochromatic light (as it should be), the intensity of the light impinging on the sample changes with the pathlength.
            \begin{itemize}
                \item In particular, if $\Delta L=n\lambda$, there is no change in intensity, but other forms see destructive interference to varying extents.
            \end{itemize}
            \item A Fourier transform then takes the monochromatic wave to a Fourier transform spectrum.
            \item The frequency resolution for the spectrometer is given by the scan distance is $\Delta\bar{\nu}=\pi/L_\text{tot}$.
            \begin{itemize}
                \item Frequency resolution given by the scan surface.
                \item For higher resolution, you need to scan farther??
            \end{itemize}
            \item We use an FTIR lamp (tungsten filament; broad bandwidth).
            \begin{itemize}
                \item Broad bandwidth is the opposite of monochromatic; thus, it is difficult to obtain repeated peaks, and a Fourier transform yields substantial intensities over a range of frequencies, as expected.
            \end{itemize}
        \end{itemize}
    \end{itemize}
    \item Measuring an FTIR spectrum.
    \begin{itemize}
        \item Take reference and sample scans (yielding $I_0$ and $I$ data) with that broadband bulb.
        \item Then take an interferogram with and without the sample.
        \item Then take an experimental spectrum, which is leveled and calculated from the previous work using $T=I/I_0$ and $A=-\log T$.
        \item What is experimental and what is mathematical manipulations here??
    \end{itemize}
    \item Wrap up.
    \begin{itemize}
        \item A spectrum originates in the time-dependent behavior of molecules driven by electromagnetic radiation.
        \item It is possible to perform experiments as a function of frequency or time.
        \begin{itemize}
            \item There are practical differences, but they encode the same information.
        \end{itemize}
        \item These are related by a Fourier transform.
        \item Fourier transform IR spectroscopy uses interferometry to relate changing optical pathlength to optical frequency.
        \begin{itemize}
            \item More on this??
        \end{itemize}
    \end{itemize}
\end{itemize}




\end{document}