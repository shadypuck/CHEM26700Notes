\documentclass[../notes.tex]{subfiles}

\pagestyle{main}
\renewcommand{\chaptermark}[1]{\markboth{\chaptername\ \thechapter\ (#1)}{}}
\setcounter{chapter}{6}

\begin{document}




\chapter{???}
\section{Office Hours (Moe)}
\begin{itemize}
    \item \marginnote{2/13:}Use chemical shift as your $x$-axis in NMR plots.
    \item Hexylamine is our substance.
    \item Interpreting NMR data files: Column E is chemical shift; column C is peak intensity. Column D is hertz (not chemical shift).
    \begin{itemize}
        \item Chemical shift is in the rightmost column.
    \end{itemize}
    \item Fitting tutorial: If solver still isn't helping (you get an error message and no convergence), divide the absolute intensities by a million. You can also do this from the get-go.
    \item $M_z$ should be calculated for each carbon; it is the absolute integral value in the all-integrals spreadsheet.
    \item Don't let the peaks overlap in the plot of multiple vertically offset $T_1$ values.
    \begin{itemize}
        \item Use 3 plots.
    \end{itemize}
    \item R outputs standard error values automatically.
    \begin{itemize}
        \item In Excel, it's much more difficult.
        \item A residuals plot is a good thing to include, but there won't be points for it. Standard error also isn't worth points. Sarah will have them upload the rubric. Sarah opened the door to email her.
        \item We need standard error for all 6 regressions.
    \end{itemize}
    \item Sarah will send a source for literature values.
\end{itemize}




\end{document}