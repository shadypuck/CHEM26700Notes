\documentclass[../notes.tex]{subfiles}

\pagestyle{main}
\renewcommand{\chaptermark}[1]{\markboth{\chaptername\ \thechapter\ (#1)}{}}
\setcounter{chapter}{5}

\begin{document}




\chapter{???}
\section{Lecture 10: Scientific Visual Communication}
\begin{itemize}
    \item \marginnote{2/7:}Long report submissions delayed until Friday.
    \begin{itemize}
        \item The content of today may be useful!
    \end{itemize}
    \item Thursday.
    \begin{itemize}
        \item Anna Wuttig on the current state of EChem.
        \item Tokmakoff will stick around after for us to chat about our reports with him.
    \end{itemize}
    \item There are many aspects beyond \emph{visual} communication, but this is one that isn't always seen, so Tokmakoff decided to focus on it today.
    \item Take all the guidelines and extra chapters seriously; they're exactly what is being graded for.
    \item We should care about communication because it's as important to our career development as anything.
    \begin{itemize}
        \item Our science must be distributed; otherwise, we're just a hobbyist.
    \end{itemize}
    \item We need to convey very complicated, quantitative information to other scientists, management, government agencies, policy makers, investors, and the general public.
    \item Reduce complex quantitative data accurately into \underline{clear, concise} messages: Data interpretation.
    \item Often, there are real requirements on content and formatting.
    \item Excellence in communication.
    \begin{itemize}
        \item Content is key, but saying it well will really level you up.
        \item It develops \emph{trust} in your methods, results, and communications.
        \item A well-communicated report and graphic can change the world, e.g., the hockey stick curve.
    \end{itemize}
    \item \textbf{Communication}: The means of exchanging information.
    \item \textbf{Medium}: Any channel of communication.
    \item Media we will discuss.
    \begin{itemize}
        \item Print (text, graphics).
        \begin{itemize}
            \item Graphics are how people digest scientific information.
        \end{itemize}
        \item Oral (in person with visual support).
        \item Never use double columns if you want transport to online.
    \end{itemize}
    \item The common starting point for all communication.
    \begin{enumerate}
        \item Audience.
        \begin{itemize}
            \item Identify; sets the objective, expectations, language, and aspects of your work to focus on.
            \item You need to know if you're talking to fellow bench scientists, or senior management.
        \end{itemize}
        \item Message.
        \begin{itemize}
            \item What are you trying to say? Just say what you need to, and get rid of the rest.
            \item When your TA or Tokmakoff reads your report, what are they going to think of my magenta line.
        \end{itemize}
        \item Media.
        \begin{itemize}
            \item What tools are at your disposal, and how are they best employed.
        \end{itemize}
    \end{enumerate}
    \item Visual presentation tips for text and graphics.
    \begin{itemize}
        \item Our goal: Communicate quantitative information clearly and concisely.
        \item Make your viewer's life easy (be consistent, define the purpose of each element, etc.).
        \item Simplicity is good; clutter is bad.
        \item Color should be chosen with a real focus in mind.
    \end{itemize}
    \item Typefaces and fonts.
    \begin{itemize}
        \item The visual representation of language. Its style should help, not interfere, with your communication.
    \end{itemize}
    \item \textbf{Typeface}: The design elements for lettering. A collection of glyphs.
    \item \textbf{Glyph}: A single representation of a character.
    \item \textbf{Font}: A variation of a typeface like size, weight, and spacing.
    \item Classes of typefaces: \textbf{Serif} and \textbf{sans serif}.
    \begin{itemize}
        \item Sans-serif is good for titles, headings, and labels.
        \item Serifs are good for presenting large amounts of text.
    \end{itemize}
    \item History of typefaces.
    \begin{itemize}
        \item Use legacy typefaces; they're still supported.
        \item "Microsoft is your friend."
        \item Computers revolutionized typography; Microsoft drove the development with proprietary stuff, which eventually caused them to lose the edge, and now there's great open-source fonts.
    \end{itemize}
    \item Typefaces for equations.
    \begin{itemize}
        \item Times New Roman and Garamond have full math support.
        \item Computer modern (\TeX) is probably still the best in terms of being able to distinguish things since it includes so many helpful flourishes.
        \begin{itemize}
            \item You're probably encountered difficulties with the lab manual (e.g., $v$ vs. $\nu$) because it's not in Computer modern.
        \end{itemize}
    \end{itemize}
    \item Tokmakoff's recommendations for formatting: Typed $8.5\times 11$ documents should have a\dots
    \begin{itemize}
        \item Single column format.
        \item $1''$ margins.
        \item 11-12 point type.
        \item $\sim 90$ characters per line including spaces (15 characters per linear inch).
        \item 4-5 lines per vertical inch.
    \end{itemize}
    \item Why worry about font size?
    \begin{itemize}
        \item Legibility vs. readability; too small impacts legibility, and too big impacts readability.
    \end{itemize}
    \item Why worry about line spacing?
    \begin{itemize}
        \item $1.5\times$ is Tokmakoff's recommendation.
        \item $2\times$ is legacy from typewriters, when single and double were the only options.
    \end{itemize}
    \item Why worry about margins?
    \begin{itemize}
        \item White space helps with clarity.
        \item Don't just insert figures; make figures break text.
    \end{itemize}
    \item Equations should be numbered.
    \item Color.
    \begin{itemize}
        \item Don't let it distract; let it help you make a cleaner presentation.
        \item Really bright colors draw the eye too much.
    \end{itemize}
    \item Scientific figures.
    \begin{itemize}
        \item Purpose: To convey quantitative information on the relationship between different physical variables with minimal effort.
        \item Each figure should convey information on exactly one topic.
        \item Again, know your audience, be aware of your medium (typed vs. oral), clarity, etc.
        \item Additional consideration for scientific reports: Often the figure is the only documentation of the data.
        \begin{itemize}
            \item If the reader wanted to analyze your data, can they read data values off the graph using the axis labels?
            \item Raw Excel sheets, other records may not be saved, so the literature report may be the only way for future scientists to reanalyze your data.
        \end{itemize}
    \end{itemize}
    \item Examples of good and bad figures.
    \begin{itemize}
        \item As you see scientific figures going forward, take note of what you like and what you don't like and learn.
        \item Tokmakoff asks for the class's feedback on his examples.
    \end{itemize}
    \item You should have 4-6 axis labels and 4-10 tick marks.
    \begin{itemize}
        \item More tick marks than labels is a good idea!
    \end{itemize}
    \item Make sure colors translate to black and white, so maybe I should vary both shapes and colors in my Birge-Sponer plot.
    \item Rowan is very picky about what Excel settings you use.
    \begin{itemize}
        \item Don't cut and paste into word; stuff gets realigned.
    \end{itemize}
    \item Tokmakoff doesn't look for units for unitless quantities (e.g., absorbance).
    \item Use a legend when there are two or more series being plotted.
    \item Caption.
    \begin{itemize}
        \item Use for report figures.
        \item It should describe what is plotted and is needed to interpret the data beyond what is in the figure itself.
        \item For data, typically quote specific experimental conditions.
    \end{itemize}
    \item Titles are only for oral presentation graphics.
    \item Don't mislead! Rescaling your axes can mislead about growth.
    \item Make everything 300 dpi.
    \item Publishers use JPG in CMYK color profile.
    \begin{itemize}
        \item Online: Use RGB color profile.
        \item Everything else is up to us.
    \end{itemize}
    \item Takeaways.
    \begin{itemize}
        \item Clarity and conciseness.
        \item White space is good!
        \item Microsoft is (mostly) your friend.
        \begin{itemize}
            \item Their templates, colors, and fonts have been professionally designed\dots with everyone in mind.
            \item Use recommended formatting, but be aware it isn't for scientists.
        \end{itemize}
        \item There are no firm rules --- just guidelines. It is an art.
    \end{itemize}
\end{itemize}




\end{document}