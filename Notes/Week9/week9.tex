\documentclass[../notes.tex]{subfiles}

\pagestyle{main}
\renewcommand{\chaptermark}[1]{\markboth{\chaptername\ \thechapter\ (#1)}{}}
\setcounter{chapter}{8}

\begin{document}




\chapter{Final Paper and Presentation}
\section{Lecture 13: Oral Communication Skills}
\begin{itemize}
    \item \marginnote{2/28:}Today: Best practices for oral presentations with graphic support.
    \item This is the last live lecture for the term.
    \item Common scientific oral presentation formats.
    \begin{itemize}
        \item Oral presentation with visual support.
        \begin{itemize}
            \item To an audience (5-200 people) with minimal disruption.
            \item The formality of your presentation scales with the size of your audience.
            \item You want to quickly, and efficiently, get a few scientific points across.
            \item Slide deck for quick presentation of quantitative results.
        \end{itemize}
        \item Poster presentations.
        \begin{itemize}
            \item Designed for one-on-one discussions.
            \item Open-ended.
        \end{itemize}
        \item Extemporaneous style for both (we are not reading from a script directly; we adapt on-the-fly to the audience's reactions and the presenters before us).
    \end{itemize}
    \item The common starting point for all communication.
    \begin{itemize}
        \item Audience, message, and media (as per Lecture 10).
        \item In our case, we know the audience, we're somewhat familiar with the media, and the message will be the challenge.
    \end{itemize}
    \item Media: Oral vs. written communication.
    \begin{itemize}
        \item The main elements are like the written report, but timing and oral delivery add challenges.
        \item The presenter's challenge: Time constraint on information presented.
        \item The audience's challenge: Can't control rate of presentation to match their comprehension, and can't re-read slides.
    \end{itemize}
    \item Planning the oral presentation.
    \begin{itemize}
        \item What part of my story can I tell in the allotted time?
        \begin{itemize}
            \item Much less oral content than the written report!
            \item Thus, make use of the visual support to communicate information quickly.
        \end{itemize}
        \item Be clear and concise.
        \begin{itemize}
            \item We have to hone the message and can't discuss stuff in as much detail as in the written report.
            \item Hint: They're not gonna be checking sig figs; we just need to convey that we know the essence of the experiment and that we've done our lab well.
        \end{itemize}
        \item Build a detailed outline of the presentation.
        \begin{itemize}
            \item Organize around a message.
        \end{itemize}
    \end{itemize}
    \item Planning the presentation.
    \begin{itemize}
        \item Tell a story --- don't give a report.
        \begin{itemize}
            \item There should be a narrative structure (with a beginning, middle, and end) to our presentation.
            \item Engage the audience, and adapt to how they're reacting.
        \end{itemize}
        \item Recall that there's maximum audience attention at the beginning and then it drops off; make a lot of use of your opening slides! Attention may pick up again toward the end.
        \item Arrange ideas in a logical sequence.
        \begin{itemize}
            \item Don't necessarily spend the most time on what took you the most time in lab! Oftentimes, that stuff is dull and you should spend zero time on it.
            \item Thus, don't necessarily go in a chronological order.
            \item Emphasize key points as you make them.
            \item Provide explicit transition points.
        \end{itemize}
    \end{itemize}
    \item Structuring the presentation.
    \begin{itemize}
        \item This will track the written report, but don't necessarily treat these as headings!
    \end{itemize}
    \item Introduction slide.
    \begin{itemize}
        \item Most important slide (everyone is paying attention, spark their interest).
        \item Introduce yourself and your collaborators.
        \item Give the big picture (introduce the central question or topic \emph{in one sentence}).
        \item Acknowledging our TA might be a good idea.
        \item Outlines aren't necessary here (maybe in longer presentations, though).
        \item See example in Tokmakoff's slides!
    \end{itemize}
    \item Background slide.
    \begin{itemize}
        \item Questions to address.
        \begin{itemize}
            \item Why is this topic worth investigating.
            \item Where this content plays a role outside of this class.
            \item Why we're interested.
        \end{itemize}
        \item DO NOT use equations in oral presentations, according to Tokmakoff's colleagues.
        \begin{itemize}
            \item So be aware! Treat it like a graphic. If the equation \emph{must} be there, you have to talk people through it like a graphic.
        \end{itemize}
        \item Go through this pretty quickly!
    \end{itemize}
    \item Experimental methods.
    \begin{itemize}
        \item Explain the links between our questions and the answer, and how our lab work got us from A to B.
        \item Explain the techniques in a bare minimal sense to get the message across.
        \item If you can find a way to eliminate technical details, then do that!
    \end{itemize}
    \item Results and discussion.
    \begin{itemize}
        \item Present the most important examples of things we measured and how it points to our conclusions.
        \item What observations did you make along the way and \emph{explain any insight you gained}.
    \end{itemize}
    \item Conclusion.
    \begin{itemize}
        \item Summarize the original question and state whether or now we answered it.
        \item Relate back to the community. What further questions are raised?
        \item Spend the first third of your talk doing something that every person in your audience will understand.
        \begin{itemize}
            \item Second third: Stuff that half the audience understands.
            \item Last third: Stuff that no one (even the speaker) understands.
            \item That's tongue and cheek, but the point is that you should end on further questions (i.e., stuff that no one understands) that you'd like to see investigated in the field.
        \end{itemize}
    \end{itemize}
    \item Q \& A.
    \begin{itemize}
        \item Anticipate questions not covered in the presentation.
        \item Bring extra (supporting) slides.
        \begin{itemize}
            \item Hopefully, we'll be able to construct answers without needing a supporting slide.
        \end{itemize}
        \item We will be asked questions at the end!
    \end{itemize}
    \item How to design effective slides.
    \begin{itemize}
        \item Limit the number of slides!
        \begin{itemize}
            \item They are for visual \emph{support}, not to give your presentation for you.
        \end{itemize}
        \item Each slide should convey the message quickly and easily.
        \begin{itemize}
            \item The average attention span per slide is 8 seconds.
            \item Simple heading.
            \item Clear statement of the message.
            \item Minimal supporting text.
        \end{itemize}
        \item Use graphics liberally.
        \begin{itemize}
            \item No clutter, though! Remember white space.
        \end{itemize}
        \item Use animation where needed.
        \begin{itemize}
            \item When we have multiple elements and it's useful to introduce material stepwise.
            \item When we have a bunch of elements, we can lead them through it one step at a time instead of having them be overloaded.
        \end{itemize}
        \item Graphs for quantitative info.
        \begin{itemize}
            \item Tables are deadly; what are you trying to compare with it if you're going to include it!
        \end{itemize}
        \item Minimize text.
        \begin{itemize}
            \item Paragraphs, complete sentences, etc. are very distracting.
        \end{itemize}
    \end{itemize}
    \item Graphics.
    \begin{itemize}
        \item The same design principles we discussed previously, but with some adjustments for the format.
        \begin{itemize}
            \item Keep them simple.
            \item Use a consistent format.
            \item Title all charts, tables, and diagrams.
            \item Use clear, explanatory labels.
            \item Everything must be legible from the back (sans serif, 24-32 pt). Tokmakoff believes that PowerPoint template defaults are too big.
        \end{itemize}
    \end{itemize}
    \item Practice the presentation.
    \begin{itemize}
        \item Rehearse!
        \item Practice several times. Then practice again.
        \begin{itemize}
            \item The first few presentations will help work out the kinks in content, organization, and delivery.
            \item Practice also assures that it doesn't sound scripted, that the content embeds in your head, and that it doesn't sound scripted.
        \end{itemize}
        \item Practice out loud with the equipment you will use.
        \item Practice with a colleague or friend for feedback. Can help catch\dots
        \begin{itemize}
            \item Content issues, typos, missing labels, and inconsistencies.
            \item Do you rock, squirm, gesture too much.
        \end{itemize}
        \item Recording yourself can also be very helpful.
        \item Time yourself --- don't go too long or too short!
        \begin{itemize}
            \item Make sure you're not a second over your time. There are plenty of conferences where they'll just yank you off.
            \item If you're too short, you'll feel like you haven't told the full story.
        \end{itemize}
        \item Think about what questions your audience will likely ask.
    \end{itemize}
    \item Delivering the presentation. On presentation day\dots
    \begin{itemize}
        \item Arrive early to gauge the room and audience.
        \begin{itemize}
            \item Be aware of seating, acoustics, and lighting.
        \end{itemize}
        \item Bring all the equipment you need. Check it and voice.
        \item Anticipate problems.
        \begin{itemize}
            \item What will you do if your equipment fails? Anticipate \emph{everything} failing.
        \end{itemize}
        \item How should you stand?
        \begin{itemize}
            \item Don't block the screen.
            \item Stand at a \ang{45} angle to the audience.
            \item Maintain eye contact with gestures to visual support.
            \item Don't turn your back to the audience.
            \item Keep your weight evenly dispersed on both feet.
        \end{itemize}
    \end{itemize}
    \item Connect with your audience.
    \begin{itemize}
        \item Put yourself in the audience's place.
        \item Use everyday language and terms.
        \item ...
    \end{itemize}
    \item Gesture and movement.
    \begin{itemize}
        \item Make nonverbal behavior deliberate; avoid extraneous motion.
        \begin{itemize}
            \item Some walking and gestures ad variety.
            \item Too much is distracting.
        \end{itemize}
        \item Use a pointer to draw attention or identify specific items on the slide.
        \begin{itemize}
            \item Don't "stir the soup" with your pointer.
        \end{itemize}
        \item When there are multiple presenters, practice positioning and handoffs with partners.
        \begin{itemize}
            \item You have so little time as it is; everything should be smooth so you don't lose any.
        \end{itemize}
    \end{itemize}
    \item Q \& A.
    \begin{itemize}
        \item Make sure you understand the question.
        \begin{itemize}
            \item Feel free to ask the questioner for clarification.
        \end{itemize}
        \item Keep your answer short and to the point.
        \begin{itemize}
            \item Don't use backup slides unless necessary.
            \item Tokmakoff may specifically ask for these!
        \end{itemize}
        \item It's ok to acknowledge gaps in your expertise if you have to.
        \begin{itemize}
            \item Explain what you do know in this case.
            \item You can say something along the lines of, "That's a great idea to try. We went sort of in that direction, but got X results and decided to stick here. Here's what we did\dots"
        \end{itemize}
    \end{itemize}
    \item Voice.
    \begin{itemize}
        \item Volume.
        \begin{itemize}
            \item Project to the back of the room and spend a lot of your eye contact on the back of the room.
        \end{itemize}
        \item Rate.
        \begin{itemize}
            \item Speak at an appropriate rate for audience comprehension.
            \item Slow down for complex or important content.
            \item Silence is great for grabbing attention back.
            \item You can keep just rolling along through minutia and then pause\dots that will draw the audience back.
        \end{itemize}
        \item Emphasis.
        \item Style.
        \item Pitch.
        \begin{itemize}
            \item Keep the pitch of your voice at a natural level.
            \item Avoid "uptalk" (the pitch of your voice going up at the end of a sentence).
        \end{itemize}
    \end{itemize}
    \item Handling anxiety.
    \begin{itemize}
        \item Remember to breathe.
        \item Practice and prepare: This helps your confidence and commits much of your presentation to memory.
        \item Write out your speech and memorize the introductory (first few) sentences. This grounds you and starts your momentum.
        \item Focus and center yourself.
        \item Don't view the situation as formal; view it as a conversation.
        \begin{itemize}
            \item You can feel like you're having a conversation with a particular person in the audience.
            \item No one is perfect --- a conversational style makes it easy to move past mistakes.
        \end{itemize}
        \item What if you freeze up or forget part of your speech?
        \begin{itemize}
            \item Pause, take some deep breaths, reorder your thoughts.
            \item If paralyzed, stop speaking and refer to your outline to reorient yourself.
        \end{itemize}
    \end{itemize}
    \item Takeaways.
    \begin{itemize}
        \item Audience, message, medium.
        \item Clarity and concision.
        \item Connect with your audience.
        \item Patience.
        \begin{itemize}
            \item Practice until you're bored to tears of practicing.
            \item A minute of speaking equals an hour of preparation.
        \end{itemize}
        \item Pace yourself.
    \end{itemize}
\end{itemize}



\section{ECHEM Full Report Notes}
\begin{itemize}
    \item \marginnote{3/2:}Plot one cycle of the CVs, one of the middle cycles. Adjust by 0.65 for SHE to mercury. Do for all three electrodes. Onset potential is where the curve starts going down.
    \item Take a current vs. voltage plot and divide the $y$-axis by area to "normalize" it and change to current density.
    \item Tin was not stable under the reaction conditions. Therefore, its surface area and chemistry would have changed throughout the run.
    \item How do you quantify catalytic activity?
    \item Normalized is more valid because it gives information about an \emph{intensive} property of the material, not the \emph{extensive} correlation.
    \item What other types of normalization are there vs. geometric normalization? Literal surface area? BET surface area?
    \begin{itemize}
        \item We can do an internet search to see what else has been done.
        \item We can integrate the peaks on a CV to see how much charge is being passed. So we can compare the catalysts based on the amount of charge being passed; charge vs. material plot.
    \end{itemize}
    \item How do we build a Tafel plot/what is the overpotential?
    \begin{itemize}
        \item The overpotential occurs at 0 on the SHE. However much negative past 0 in the converted data is our overpotential. But instead of plotting the negative log CA data vs. these "raw" negative overpotentials, we'll plot vs. the "volts past zero."
    \end{itemize}
    \item The $\pH$ of \SI{0.1}{\molar} \ce{H2SO4} is calculated by assuming 100\% dissociation, i.e., $[\ce{H3O+}]=\SI{0.2}{\mole\per\liter}$. Thus, $\pH=-\log(0.2)\approx 0.7$.
    \begin{itemize}
        \item Onset potential of the \ce{Pt} CV is about \SI{0.05}{\volt}.
        \item At this $\pH$ and potential, it appears that we're just crossing the (a) line in the Pourbaix diagram.
    \end{itemize}
\end{itemize}




\end{document}